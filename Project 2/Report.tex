\documentclass{article}
\usepackage{ctex}
\usepackage{listings}
\usepackage{xcolor}

\title{深度学习与神经网络第二次课程项目}
\author{王逸群 19307110397}
\date{2022.4.9}

\definecolor{gray}{rgb}{0.99,0.99,0.99}

\lstdefinestyle{mystyle}{
	basicstyle=\footnotesize,
	backgroundcolor=\color{gray},
	numbers=left
}

\lstset{style=mystyle}

\begin{document}
	
\maketitle

\section{神经网络}

\subsection{初始架构}

本项目使用CIFAR-10数据集,
其中包含60000张$32\times32$的彩色图片,
被平均分为10类:飞机、汽车、鸟、猫、鹿、狗、青蛙、马、船、货车。

参考VGG网络架构,
基于pytorch框架,
设计神经网络初始架构存储于\verb|Code/nn.py|。
具体内容如下:

\begin{lstlisting}[language=Python]
class NN(nn.Module):
    def __init__(self, in_channels = 3, hidden_channels = (16, 32), 
                 hidden_neurons = (128, 128), num_classes = 10):
        super().__init__()
        self.hidden_channels = hidden_channels

        self.extractor = nn.Sequential(
            # stage 1
            nn.Conv2d(in_channels = in_channels,
            out_channels = hidden_channels[0],
            kernel_size = 3, padding = 1),
            nn.ReLU(),
            nn.MaxPool2d(kernel_size = 2, stride = 2),

            # stage 2
            nn.Conv2d(in_channels = hidden_channels[0],
            out_channels = hidden_channels[1],
            kernel_size = 3, padding = 1),
            nn.ReLU(),
            nn.MaxPool2d(kernel_size = 2, stride = 2))

        self.classifier = nn.Sequential(
            nn.Linear(hidden_channels[1] * 8 * 8, hidden_neurons[0]),
            nn.ReLU(),
            nn.Linear(hidden_neurons[0], hidden_neurons[1]),
            nn.ReLU(),
            nn.Linear(hidden_neurons[1], num_classes))

    def forward(self, inputs):
        hidden = self.extractor(inputs)
        outputs = \
            self.classifier(hidden.view(-1,
                                        self.hidden_channels[1] * 8 * 8))
        return outputs
\end{lstlisting}

\subsection{参数调整}

\subsubsection{神经元数量}

\subsubsection{损失函数}

\subsubsection{正则化}

\subsubsection{激活函数}

\subsubsection{优化器}

\subsubsection{批归一化}

\subsubsection{丢弃法}

\subsubsection{残差连接}

\subsection{可视化}

\subsubsection{卷积核可视化}

\subsubsection{损失函数曲线}

\section{批归一化}
	
\end{document}